\documentclass[12pt]{article}

\usepackage{times}
\usepackage{amsmath}
\usepackage{accents}
\usepackage{geometry}
\geometry{a4paper}
\geometry{margin=20mm}


%%%%%%%%%%%%%%%%%%%%%%%%%%%%%%%%%%%%%%%%%%%%%%%%%%%%%%%%%%%%%%%%%%%%%%%%%%%%%%%%
%%%%%%%%%%%%%%%%%%%%%%%%%%%%%%%%%%%%%%%%%%%%%%%%%%%%%%%%%%%%%%%%%%%%%%%%%%%%%%%%
%%%%%%%%%%%%%%%%%%%%%%%%%%%%%%%%%%%%%%%%%%%%%%%%%%%%%%%%%%%%%%%%%%%%%%%%%%%%%%%%
\begin{document}


Most commonly used equations of state or free energy models are
explicit in the temperature and species densities; they typically only
implicitly depend on the temperature.  In this case, the chemical
potential is given as
\begin{equation}
\mu_\alpha(T,{\bf c}) = \mu_\alpha^{\rm ig}(T,{\bf c})
+ \mu_\alpha^{\rm res}(T,{\bf c})
\end{equation}
Unfortunately, this is not convenient for phase equilibria
calculations; however, we can obtain an expression for the chemical
potential of the system at a fixed temperature, pressure, and
composition as:
\begin{align*}
\mu_\alpha(T,p,{\bf x}) &= \mu_\alpha^{\rm ig}(T,{\bf c})
+ \mu_\alpha^{\rm res}(T,{\bf c})
\\
&= \mu_\alpha^{\rm ig}(T,p) + RT\ln x_\alpha
- RT\ln Z(T,{\bf c})
+ \mu_\alpha^{\rm res}(T,{\bf c})
\end{align*}
where $Z=pV/(RT)$ is the compressibility factor.


For an activity coefficient model, the chemical potential is given as
\begin{align}
\mu_\alpha = \mu_\alpha^{\rm l}(T,p)
+ RT \ln x_\alpha \gamma_\alpha
\end{align}


\section{Notes on the reference state}

Each molecule (component) should contain a member function that
returns the chemical potential (molar Gibbs free energy) of the pure
species at a specified temperature, pressure, and phase.  Possible
states are the ideal gas state, the liquid state, or a crystalline
state.
\begin{align}
\mu(T,p) &= \mu(T_0,p_0) 
+ \int_{P_0}^{P} dP' \left(\frac{\partial\mu}{\partial p}\right)_T 
+ \int_{T_0}^{T} dT' 
  \left(\frac{\partial \mu}{\partial T}\right)_p 
\nonumber
%\\
%&= \mu(T_0,p_0) 
%+ \int_{p_0}^{p} dp' V(T,p')
%- \int_{T_0}^{T} dT' S(T,p_0)
%\nonumber
%\\
%&= \mu(T_0,p_0) 
%+ \int_{p_0}^{p} dp' V(T,p')
%- \int_{T_0}^{T} dT' \left[
%    S(T_0,p_0) + \int_{T_0}^{T'} dT'' \frac{C_p(T'',p_0)}{T''} \right]
%\nonumber
\\
\begin{split}
&= \mu(T_0,p_0) 
+ \int_{P_0}^{P} dP' V(T,p')
+ \left(\frac{T}{T_0}-1\right) [\mu(T_0,p_0)-H(T_0,p_0)]
\\ & \qquad
- \int_{T_0}^{T} dT' \left(\frac{T}{T'}-1\right) C_p(T',p_0)
\end{split}
\end{align}


In the case of an ideal gas reference state, this becomes
\begin{equation}
\begin{split}
\mu^{\rm ig}(T,p) &= \mu^{\rm ig}(T_0,p_0) 
+ RT \ln \frac{p}{p_0}
+ \left(\frac{T}{T_0}-1\right) [G^{\rm ig}(T_0,p_0)-H^{\rm ig}(T_0,p_0)]
\\ & \qquad
- \int_{T_0}^{T} dT' \left(\frac{T}{T'}-1\right) C_p^{\rm ig}(T',p_0)
\end{split}
\end{equation}


The NIST database should provide us with information for a wide
variety of components in the ideal gas state.


In the case of a liquid where we are missing the information of the
reference state, we can estimate this from information from the ideal
gas chemical potential and its vapor-liquid equilibria:
\begin{align}
\mu^{\rm l}(T,p) &= \mu^{\rm l}(T,p^{\rm vap}(T)) 
+ [\mu^{\rm l}(T,p) - \mu^{\rm l}(T,p^{\rm vap}(T))]
\nonumber
\\
&= \mu^{\rm g}(T,p^{\rm vap}(T)) 
+ \int_{p^{\rm vap}(T)}^p dp' V^{\rm l}(T,p')
\nonumber
\\
&= \mu^{\rm ig}(T,p^{\rm vap}(T)) 
+ [\mu^{\rm g}(T,p^{\rm vap}(T))-\mu^{\rm ig}(T,p^{\rm vap}(T))] 
+ \int_{p^{\rm vap}(T)}^p dp' V^{\rm l}(T,p')
\nonumber
\\
\begin{split}
&= \mu^{\rm ig}(T,p) + RT\ln\frac{p^{\rm vap}(T) }{p}
+ [\mu^{\rm g}(T,p^{\rm vap}(T))-\mu^{\rm ig}(T,p^{\rm vap}(T))] 
\\ & \qquad
+ \int_{p^{\rm vap}(T)}^p dp' V^{\rm l}(T,p')
\end{split}
\end{align}


%%%%%%%%%%%%%%%%%%%%%%%%%%%%%%%%%%%%%%%%%%%%%%%%%%%%%%%%%%%%%%%%%%%%%%%%%%%%%%%%
%%%%%%%%%%%%%%%%%%%%%%%%%%%%%%%%%%%%%%%%%%%%%%%%%%%%%%%%%%%%%%%%%%%%%%%%%%%%%%%%
\section{Equilibrium}


$N_\alpha^{(A)}$ is the moles of component $\alpha$ in phase $A$.

The total Gibbs free energy $\underaccent{\bar}{G}$ of the system
is given by
\begin{align*}
\underaccent{\bar}{G} &= \sum_{\alpha A} N_\alpha^{(A)} \mu_\alpha^{(A)} 
\end{align*}

Component mole balances
\begin{equation}
\sum_A  N_\alpha^{(A)} = N_\alpha
\end{equation}

\begin{align*}
d\underaccent{\bar}{G} &= \sum_{\alpha A}  \mu_\alpha^{(A)}  dN_\alpha^{(A)}
\\
d \underaccent{\bar}{F} 
&= \sum_{\alpha A}  (\mu_\alpha^{(A)}- \lambda_\alpha) dN_\alpha^{(A)}
\end{align*}



%%%%%%%%%%%%%%%%%%%%%%%%%%%%%%%%%%%%%%%%%%%%%%%%%%%%%%%%%%%%%%%%%%%%%%%%%%%%%%%%
%%%%%%%%%%%%%%%%%%%%%%%%%%%%%%%%%%%%%%%%%%%%%%%%%%%%%%%%%%%%%%%%%%%%%%%%%%%%%%%%
%%%%%%%%%%%%%%%%%%%%%%%%%%%%%%%%%%%%%%%%%%%%%%%%%%%%%%%%%%%%%%%%%%%%%%%%%%%%%%%%
\end{document}
